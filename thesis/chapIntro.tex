\chapter{Introduction}
Public purchases constitute a considerable portion of the government budget. Taxpayers expect public purchases to be transparent, efficient in cost and effective in producing public goods. The existence of competitive markets for each of the types of products purchased by the government contributes to achieve these objectives by preventing rent seeking, difficulting collusive arrangements, and favouring innovation. Competitiveness is widely accepted to be negatively affected by the existence of artificial entry barriers, like regulation or collusion. However, a more complicated case arises when the entry barriers arise from the human and capital advantages gained by early arrivers at the market. In this case, the human and organizational capital achieved through experience renders early firms more efficient, but at the same time can curb competition in the future by reducing entry.

This thesis investigates whether past experience confers public contractors an advantage when bidding for public construction contracts. The importance of analyzing this effect is that the advantages gained by "learning by doing" are likely to affect competitiveness in the market, and ultimately quality and prices obtained by the government. Altough the scope of the market analyzed makes it i) heterogenous in terms of products and providers, and ii) of considerable size, so that it would be very unlikely to reach a monopoly state, the question remains as to the degree of impact that this effect have on competition.

We employ a dataset of more than 30,000 contracts for public construction projects, totalling approximately 110,000 individual firm bids across 10 years, to study the treatment effect of experience on future bidding outcomes. The sample is obtained from government auctions developed in Chile, where the importance of sound procurement processes and results have been highlighted by key laws passed in the last 15 years aimed at increasing transparency and efficiency and have created strict information reporting requirements for government units. For most of the government units included in the sample, the data is comperehensive with respect to the coverage of all the auctions held for projects of the category of interest.

We choose to examine specificially the construction sector because of several reasons. First, construction projects are highly differentiated in comparison to other types of goods procured by the government, which makes them more complex and expectedly more difficult for newcomers. Second, several types of the projects procured by the government in this sector are not developed in the provate sector, such as roads and parks. Finally, given the magnitude of the spending required to perform construction projects, they are usually one of the main focus in study. Later, in the wake of the pandemic produced by COVID-19, one of the trends observed has been to at least propose an increment in the budget for these types of projects in order to stimulate the economy.

 The research objective is to identify the treatment effect of experience on the outcomes of firms in the market of public construction projects. We consider as outcomes of interest the share of contracts won by each firm out of total contracts bid for, in subsequent time periods of around two years each. The empirical design consists on producing several "slices" in time, each composed by a period in which we compute experience and a subsequent period where we compute the outcomes, for each firm. We employ these slices to perform regressions between different computations of experience as the treatment variable and winning shares of firms in the market as the outcome variable. Our 10-year data allows us to produce analysis at several points in time which help to prevent confounding noise from temporal market trends.

Our OLS results show that the existence of positive experience is associated with an increase of around seven percentage points in mean future winning shares, which equals around 20\% of the dependent variable's standard deviation and almost 30\% of its mean. Every extra contract won in the past period is associated with around an extra 1 percentage points in winning share. All the key estimates are significant at $p<0.01$ and with low standard errors. We find however high heterogeneity in outcomes and low $R^2$ in our regressions.

 Because experience is likely to be endogenous with unobserved cost factors, specific to each firm, we employ external variation on experience to produce consistent estimates of the treatment effect. Our identification strategy employs closely won contracts as the source of random variation in experience levels, arguing that they cannot be attributed to cost advantages. We define "close wins" by two alternative strategies. The first one labels a win as close if price was more than half of the awarding criteria and winning bids were close to other competitors' bids. The second alternative labels a win as close if all firms participating in the auction had a similar rank, which we compute for every firm at every point in time via a multiplayer ELO algorithm. Our resulting IV estimates remain close to OLS counterparts: between 7.3 and 7.8 percentage points for an indicator of positive experience as tretament and between .8 and 1.2 percentage points for total experience.

 We perform robustness analysis on several of the parameters employed either to construct our analysis sample or in the identification strategy, especially the ones related to the definition of a close win, like thresholds of closeness between bids and firm's ranks. The results show robustness to most of the parameters employed, altough we lose power to obtain significant estimates at very high thresholds for the instruments, especially for the price IV strategy.

 Next, we present and investigate two hypothesis regarding the underlying mechanisms that could explain the improved outcomes for firms that acquire experience: improvements in cost measures and quality improvements in proposals. We test the first hypothesis by analyzing the evolution of firm bids among firms with different levels of experience. We find evidence that confirms that more experienced firms submit lower bids: the treatment effect of positive experience on bids is to reduce standarized bid amounts (i.e. the quotient of raw bids on government estimates of the cost of the project) by around four percentage points. The effect is relevant considering that the lowest and second lowest bid is almost seven percentage points.

Regarding the second hypothesis, we test it by analyzing the rate of acceptance of firms' proposals in the first stage of the awarding process, which controls that the proposals fulfill a set of basic non-economic, mostly formal criteria.  Employing similar identification techniques as before, we find that the treatment effect of experience is to increase in around ten percentage points the future mean acceptance rates, which is around a third of the standard deviation of the outcome variable in the analysis sample.

In chapter  we discuss the impacts of the estimates found, discuss the strengths and limitations of the findings and point towards remaining questions. we also discuss the heterogeneity of outcomes and possible effects in the competitiveness in the market. Chapter concludes.

The structure is as follows. Chapter 2 presents the Institutional Context of public purchases, especially for construction projects. Chapter three details the source and characteristics of the data. Chapter four contains our main analysis of the effect of experience on outcomes. Chapter five studies the possible operational ways in which experience can increase the advantages of a firm in the market. Chapter six presents a discussion of the results obtained and chapter seven concludes.
