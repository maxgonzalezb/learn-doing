\chapter{Introduction}
Public purchases constitute a considerable portion of the government budget. Taxpayers expect punlic purchases to be transparent, efficient in cost and effective in producing public goods. A condition which usually helps to achieve these objectives is the existence of competitive markets in each of the types of products purchased by the government. This helps to prevent rent seeking, leaves out collusive arrangements, and favours innovation to obtain the best quality products at the lowest possible cost.

This thesis investigates whether experience confers public contractors an advantage when bidding for public construction contracts. The importance of analyzing this effect is that advantages gained by "learning by doing" are likely to affect competitiveness in the market, and ultimately quality and prices obtained by the government. Since intial winners gain advantages that compound over time to increase their future likely of winning public auctions. The market analyzed is highly heterogenous and of size enough that it would be very unlikely to reach this state, altough it remains the question of whether we observe reduced competition or entry barriers in the presence of higher returns to experience.

We employ a dataset of more than 30,000 contracts for public construction projects, totalling approximately 110,000 individual firm bids, to study the treatment effect of experience on future bidding outcomes. Our outcomes of interest are the share of contracts won by each firm out of total contracts bid for, which we relate to the success in the firms in the market. Our sample is obtained from public auctions developed in Chile, where the importance of sound procurement processes and results have been highlighted by key laws passed in the last 15 years aimed at increasing transparency and efficiency.

We choose to examine specificially the construction sector because of several reasons. First, construction projects are highly differentiated in comparison to other types of goods procured by the government, which makes them more complex and expectedly more difficult for newcomers. Second, several types of the projects procured by the government in this sector are not developed in the provate sector, such as roads and parks. Finally, given the magnitude of the spending required to perform construction projects, they are usually one of the main focus in study. Later, in the wake of the pandemic produced by COVID-19, one of the trends observed has been to at least propose an increment in the budget for these types of projects in order to stimulate the economy.

Our data covers ten years of public purchases. The empirical design relies on producing several "slices" in time, each composed by a period in which we compute experience and a subsequent period where we compute the outcomes for firms. We use these slices to perform regressions between different measures of experience as the treatment variable and winning shares of firms  in the market as the outcome variable. Our 10-year data allows us to produce analysis at several points in time which help to prevent confounding noise from temporal market trends.

Our OLS results show treatment effects that range between for the binary experience (i.e the treatment variable is having won at least one past contract) and between for linear experience (the treatment is the number of past contracts won). All treatment p-values are significant at $p<0.01$. We find however great heterogeneity in the outcomes signaled by low $R^2$ in our regressions.

 Because experience is likely to be endogenous with unobserved cost factors specific to each firm, we require external variation on experience to produce consistent estimates of the treatment effect. Our identification strategy employs closely won contracts as the source of random variation, arguing that they cannot be attributed to cost advantages. We define "close wins" by two alternative strategies. The first one labels a win as close if price was more than half of the awarding criteria and winning bids were close to other competitors' bids. The second alternative labels a win as close if all firms participating in the auction had a similar rank, which we define via a multiplayer ELO algorithm. Our resulting IV estimates remain close to OLS counterparts: between for the binary indicator and for the linear version.

 We perform robustness analysis on several of the parameters employed either to sample construction or identification strategy. The results show robustness to most of the parameters employed, altough we lose power to obtain significant estimates at very high thresholds for the instruments, especially for the price IV strategy.

We present and investigate some hypothesis regarding the underlying mechanisms that could explain the improved outcomes for firms that acquire experience. Explore some hypothesis, lower bids, improved quality. Our two main candidates are cost efficiency improvements and quality improvements. We test the first by analyzing the evolution of firm bids and the second by analyzing the rate of acceptance of firms bid past a formal revision stage, where proposals that do not fulfill a set of basic (non-economic) requirements are discarded. We find evidence that confirms that more experienced firms submit lower bids and also that more experienced firms have higher acceptance rate of their proposals than unexperienced ones. Employing similar identification techniques as before, we find that IV estimates range between  and .

Finally, we replicate our former analysis for other types of projects procured by the government. We find that.

In the discussion section we examine the magnitudes of the estimates found  relative to contracts, employing as a reference contracts that rewards experience explicitly and the different estimates found. Moreover, we discuss the heterogeneity of outcomes and possible effects in the competitiveness in the market.

The structure is as follows. Chapter 2 presents the Institutional Context of public purchases, especially for construction projects. Chapter three details the source and characteristics of the data. Chapter four contains our main analysis of the effect of experience on outcomes. Chapter five advances some possible mechanisms to explain the results obtained. Chapter six presents a discussion of the results obtained and chapter seven concludes.
