\chapter{Introduction}
Public purchases constitute a considerable part of the government' expenditures budget. For government officials, procuring public goods with transparency, efficency and competitevenss should be a key concern. This is especially true for construction projects,
since problems at the procurement stage are likely to cause big negative effects in the delivery phase. 

The importance of analyzing the treatment effect of experience is that advantages gained by "learning by doing" are likely to affect competitiveness in the market, and ultimately quality and prices. The theoretical analysis by and showed that learn by doing can steer a market towards monopoly, since intial winners gain advantages that compound over time to increase their future likely of winning public auctions. The market analyzed is highly heterogenous and of size enough that it would be very unlikely to reach this state, altough it remains the question of whether we observe reduced competition or entry barriers in the presence of higher returns to experience.

This thesis investigates whether experience confers public contractors an advantage when bidding for public construction contracts. We employ a dataset of more than 30,000 contracts for public construction projects, totalling approximately 110,000 individual firm bids, to study the treatment effect of past experience on future bidding outcomes. Our outcomes of interest are the share of contracts won by each firm out of total contracts bid for.

The empirical designs relies on producing several "slices" in time, each composed by a period in which we compute experience and a subsequent period where we compute the outcomes for firms. We use these slices to perform regressions between different measures of experience as the treatment variable and winning shares as the outcome variable. Our 10-year data allows us to produce analysis at several points in time which help to prevent confounding noise from temporal market trends.

Our results OLS results show coefficients(treatment effects) that range between for the binary experience (i.e the treatment variable is having won at least one past contract) and between for linear experience (the treatment is the number of past contracts won). All treatment p-values are significant at $p<0.01$.

 Because experience is likely to be endogenous to unobserved cost factors specific to each firm, we require external variation on experience to produce consistent estimates of the treatment effect. Our identification strategy employs closely won contracts as the source of random variation, arguing that they cannot be attributed to cost advantages. We define "close wins" by two alternative strategies. The first one labels a win as close if price was more than half of the awarding criteria and winning bids were close to other competitors' bids. The second alternative labels a win as close if all firms participating in the auction had a similar rank, which we define via a multiplayer ELO algorithm. Our resulting IV estimates remain close to OLS counterparts: between for the binary indicator and for the linear version.

 We perform robustness analysis on several of the parameters employed either to sample construction or identification strategy. The results show robustness to most of the parameters employed, altough we lose power to obtain significant estimates at very high thresholds for the instruments, especially for the price IV strategy.

We present some hypothesis regarding the underlying mechanisms that could explain the improved outcomes. Explore some hypothesis, lower bids, improved quality. We find some evidence points to quality. This would also explain close iVS.

Finally, we replicate our former analysis for other types of projects procured by the government. We find that.

The structure is as follows. Chapter 2 presents the Institutional Context of public purchases, especially for construction projects. Chapter three details the source and characteristics of the data. Chapter four contains our main analysis of the effect of experience on outcomes. Chapter five advances some possible mechanisms to explain the results obtained. Chapter six presents a discussion of the results obtained and chapter seven concludes.
