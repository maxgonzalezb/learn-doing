\chapter{Introduction}
Public purchases constitute a sizable proportion of the government's budget. Taxpayers expect public purchases to be transparent, efficient in cost and effective in the production public goods. The existence of competitive markets for each of the types of products purchased by the government is seen as a necessary condition for efficient procurement. Usually, competitiveness is accepted to be negatively affected by the existence of artificial entry barriers, like regulation or collusion. However, a more complicated case arises if participants in the market can gain competitive advantages through experience ("learning by doing"). In this case, the human and organizational capital acquired by performing works can improve a firm's competitivity and overall short term social welfare, but at the same time it can curb future competition in the market by reducing entry or making it difficult for new entrants to succeed.

This thesis investigates whether past experience causally improves future outcomes for contractors in the market for public construction contracts.  We consider as outcomes of interest the share of contracts won by each firm, out of total contracts bid for, in subsequent time periods. The treatment variable considered is experience, measured as past wins in the market. Several ways of  computing experience are employed (i.e. rolling, cumulative) as well as two functional forms for it(binary indicator and continous total).

The empirical design consists on producing several "slices" in time, each composed by a period in which we compute experience and a subsequent period where we compute the outcomes. We employ these slices to perform regressions between different measures of experience as the treatment variable and winning shares of firms as the outcome variable. We add time controls  to prevent confounding noise from temporal market trends.

We employ a dataset of more than 43,000 procurement auctions of public construction projects in Chile, totaling approximately 150,000 individual firm bids across 11 years, to study the treatment effect of experience on future bidding outcomes. The sample contains all geographic regions and a collection of more than 900 individual buyers (government units) and 15,000 individual sellers (firms). For most of the government units included in the sample, the data is comprehensive in its coverage of auctions held for projects of the construction. The wide scope of the data is achieved because of key laws passed in the last 15 years in Chile aimed at increasing transparency and efficiency, which have created information reporting requirements for government units regarding public purchases.

%The importance of analyzing this effect is that the advantages gained by "learning by doing" are likely to affect competitiveness in the market, and ultimately quality and prices obtained by the government. Although the scope of the market analyzed makes it i) heterogenous in terms of products and providers, and ii) of considerable size, so that it would be very unlikely to reach a monopoly state, the question remains as to the degree of impact that this effect have on competition.

The OLS results of regressions on outcomes on experience show that the existence of positive experience is associated with an increase of between 6.1 and 7.4 percentage points in mean future winning shares, which equals around 20\% of the dependent variable's standard deviation and almost a third of its mean. Every extra contract won in the past period is associated with between 1.0 and 2.7 extra percentage points in winning shares. All the key estimates are significant at $p<0.01$ and with low standard errors. We find however high heterogeneity in outcomes and low $R^2$ in our regressions.

The research objective is to identify the treatment effect of experience on the outcomes of firms in the market of public construction projects, but because experience is likely to be endogenous with unobserved cost factors, specific to each firm, the OLS estimates are not likely to be consistent. We employ external variation on experience to produce consistent estimates of the treatment effect. Our identification strategy employs closely won contracts as the source of random variation in experience levels, arguing that they cannot be attributed to unobserved cost advantages. We define "close wins" by two alternative strategies. The first one labels a win as close if price was more than half of the awarding criteria and winning bids were close to other competitors' bids. The second alternative labels a win as close if all firms participating in the auction had a similar rank, which we compute for every firm at every point in time via a multiplayer Elo algorithm. We argue that the empirical strategy identifies the Local Average Treatment Effect for the complier subsample of firms.

The resulting IV estimates are close but higher than OLS counterparts: between 6.3 and 8.4 percentage points for an indicator of positive experience as treatment and between .7 and 3.2 percentage points for continuous experience. We perform robustness analysis on several of the parameters employed either to construct our analysis sample or in the identification strategy, especially the ones related to the definition of a close win, such as the thresholds of closeness between bids and the allowed bandwidth for firm's ranks. The results show robustness to most of the parameters employed, although we lose power to obtain significant estimates at very high thresholds for the price IV strategy.

 Next, we present and investigate two hypothesis regarding the underlying mechanisms that could explain the improved outcomes for firms that acquire experience: improvements in cost measures and quality improvements in proposals. We test the first hypothesis by analyzing the evolution of firm bids' amounts among firms with different levels of experience. We find evidence that confirms that more experienced firms submit lower bids: the treatment effect of positive experience on bids is to reduce standardized bid amounts (i.e. the quotient of monetary bids on government estimates of the cost of the project) by around four percentage points. The effect is relevant considering that the average difference between lowest and second lowest bid is around nine percentage points.

Regarding the second hypothesis, we test it by analyzing the rate of acceptance of firms' proposals in the first stage of the awarding process, which controls that the proposals fulfill a set of basic non-economic, mostly formal criteria.  Employing similar identification techniques as before, we find that the treatment effect of binary experience is to increase in around ten percentage points the future mean acceptance rates, which is around a third of the standard deviation of the outcome variable in the analysis sample.

The Discussion chapter reviews the magnitude of the estimates found, analyzes the strengths and limitations of the findings and examine the econometric interpretation of the estimates. We also discuss the heterogeneity of outcomes and possible effects in the competitiveness in the market.

We chose to examine specifically the construction sector because of several reasons. First, construction projects are more differentiated in comparison to other types of goods procured by the government, which makes them more complex and expectedly more difficult for newcomers. Second, several types of the projects procured by the government in this sector are not developed in the private sector, such as roads and parks. Finally, given the magnitude of the spending required to perform construction projects, they are usually one of the main focus in the study of public efficiency. Moreover, in the aftermath of the pandemic produced by COVID-19, one of the trends observed across countries has been to propose increases in the budget for these types of projects.

The structure is as follows. Chapter two presents the relevant literature. Chapter three describes the institutional context of public purchases, especially for construction projects. Chapter four  details the source and characteristics of the data. Chapter five contains our main analysis of the effect of experience on outcomes. Chapter six studies the possible operational ways in which experience can increase the advantages of a firm in the market. Chapter seven presents a discussion of the results obtained and chapter eight concludes.
