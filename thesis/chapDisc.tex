\chapter{Discussion}

In this chapter we analyze the results observed, the implications and the confiability of our results.

\section{Experience and Outcomes}
Improved outcomes for firms with experience were successfully showed to be superior to the outcomes of firms with no experience. We found an increase in winning shares of between seven and nine percetage points for the treatment with any experience, and between .05 and 2 percentage points for extra unit of experience. The magnitude of these effects seems to be relevant for the path of a firm, since the mean of success when bidding is around 22\%. That is, experience can render almost a 30\% improvement in future outcomes, measured as contracts won out of contracts bid for.

Our instrumental variables approaches were very different. The first relied on close wins identified by close competition on price, while the second relied in finding contests between "similar firms", via a ranking algorithm. Altough they were different, they rendered similar results. The advantage of the price srategy is that it is more interpretable, however, the conditions imposed were so stringent as to decrease very highly the sample avalaible, which increased standard errors. The rank srategy is less interpretable, but it should control for any unobservable factor, not just price advantages. Overall, the key weakness of the rank strategy is the necessity of an adjustment period for newcomers, so ranks for first entrants (which are the most important) are less precise than firms which have been long in the market. Also, the bigger amount of parameters required made the srtategy less robust.

One interesting point is that IV strategies render almost always higher values or close to the OLS, when the original hypothesis was that this would be higher. Two points can be mentioned to explain this. First, the experience measure (contracts won, in any of its forms) is noisy measure of experience, since actual learning or improvements depend highly on the size of the contract, type of project, etc. Thus, since the instruments are positively correlated with this noisy measure, we find higher coefficients. If this is the case, the bias is small. The second possibility is that there is a selection effect which takes out firms from the market when they have not experience. Other firms self select, actual efect is higher.

A limitation of the analysis is that we are only able to identify the Local Average Treatment, which given our restrictive instrument definition, is applicable to only a small part of our observations (between 2\% and 15\%, depending on the IV).

Identification: General tendency is probable, but must be emphasisez the average nature of the treatment effect. The effect of experience is very different for different types of projects. Could experiment with weighte, non monetary items. One point that requires is self-selection effects. Firms should bid for contracts in which they expect to win, given bid preparation costs. One explanation is thet firms are miopic. Atributhe the heterogeneity and prevent identifiaction of caussal.

The comparison of estimates for contracts that explicitly rewarded experience is relevant because it shows that the implicit effect of experience on outcomes is at least in the same order of magnitude as the explicit effect. Given this, policymakers might prefer to only employ experience as a prequalification method, since it seems to largely operate in the case of no explicit reward for experience.

An important dimension that the analysis does not capture is the effect of experience in entry and exit. The effect of experience on entry and exit is uncertain. If the environment is too hostile, for example, experience miught induce exit. However, if firms perceive return to experience, we should see increased exit among non-experienced firms. IN the latter, the treatment effect of experience underestimates the true returns, since firsm in the market survive precisely because of the experience, so only the best reamain to bid.
along analysis does not capture are exits from the market. We briefly show in plot how exits disaggregated in terms of \% of firms that exit with and withoutt experience per year.

Finally, experience is correlated with only time in the market. We found (on the appendix) that the effect of only being in the market is OLS m which is considerable as well. Especially for bid preparation.


\section{Mechanisms}

The operative mechanisms section's objective was to test hypothesis regarding the improvements caused by experience. Two possibilities were examined: an improvement in price, measured by the level of standarized bids submitted, and the level of quality, measured by the rate of acceptance of offers.

Firstly, the hypoithesis that experience causes reduction in cost measures was tested. It was found that bids of firms with any kind of experience were between 10 and 12 percentage points lower than those that did not have any. The average difference between lowest and second lowest firms is around, so the impact can be significant if there is a binary reward to the lowest bid submitted. Is this difference truly useful to win more contracts? it could be argued that given the wide amount of criteria the effect would be negligible. However, a quick regression of winning outcome (0-1) on standarized bid, with some fixed effects (see Appendix) shows that for every less percentage point on bid amounts, winning probabilities increase by around .

Additionally, the result that first entrants bid more aggresively than rest of the players was in line with . Notably, the net effect of experience and first entry shows that an experienced firm still submits lower bids on average than first entrants.

The next hypothesis examined was that experience improves the quality of the proposals that a firm submits for auctions, employing as a measure of quality the acceptance rate of its proposals in a formal check stage. We found that firms with stricly positive experience have acceptance rates that are around ten percentage points higher than firms with no experience. This effect is relevant considering that the average rate of acceptance is around, so the effect of experience drives acceptance rates close to 90\%.

It could be argued that the effect corresponds only to an adaptation experienced naturally after participating in the first "trial" auctions. Howevever, the analogous effect of $bidding experience$ on outcomes is only percentage points (details on the Appendix). While there seems to be a component of the effect realted to "knowing the market", the effect of experience goes above and beyond this phenomena.

The difference is especially striking given the already high levels, whhich make expected smaller.

Again, selection effects could play a part.

\section{Data and others}
Data contains important variables.

More transparency and analysis would be achieved with partial or even total scores.
This way, it is extremely  difficult.

Possible alternative strategy- follow each firm.

Tradeoff in singular types of markets vs. this wide measures. Only attempt singular in closed markets such housing and public works.

Difficulty of achieving the experience criteria, which reuqired careful usage of API and crawlers to find, extract and parse the information.

\section{Implications for the market}
reduce artificial barriers.
increase the technical requrements.
the market rewards those wityh experience anyways. If not, possible good.

\chapter{Conclusion}
Hard to model all auction environment. how much repeats?
Could we define different markets??->filter only for biggest contracts.
