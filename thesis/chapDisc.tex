\chapter{Discussion}

\section{Experience and Outcomes}
Winning rates of firms with experience were successfully showed to be superior to the winning rates  of firms with no experience. We found an increase of between seven and nine percentage points in winning rates for the treatment with any experience, and between .05 and 2 percentage points for every extra unit of experience. The magnitude of these effects seems to be relevant for the overall outcomes of a firm, since the mean rate of success when bidding is around 22\%. That means experience can render almost a 30\% improvement in future outcomes, measured as contracts won out of contracts bid for.

Our instrumental variables approaches to obtain consistent estimates were very different between them but rendered similar results. The first relied on close wins identified by close competition on price, while the second relied in finding contests between "similar firms", via a ranking algorithm. The advantage of the price strategy is that it is more interpretable, however, the conditions imposed were so stringent that the resulting "complier" sample was very small. The rank strategy is less interpretable, but theoretically it should control for any unobservable factor that influences firm's outcomes, not just cost advantages. The major weakness of the rank strategy is the necessity of an adjustment period for newcomers, so ranks for first entrants (which are the most important ones) are less precise than those of firms which have been longer in the market. Additionally, having an additional set of parameter (points awarded for win and lose) can make the strategy less  robust. 

Interestingly, the IV strategies rendered almost always higher estimates than the OLS, when the original hypothesis was that an upwards bias would be found. Two points can be mentioned to explain this. First, the experience measure (contracts won, in any of its forms) is a noisy measure of experience, since actual learning or improvements depends highly on the size of the contract, type of project, etc. Then, there is an attenuation effect in the OLS estimates.

The second possibility is that there is a selection effect which takes out firms from the market when they are unable to gain experience. In principle, the effect of experience on entry and exit is uncertain. If the environment is too hostile, for example, and firms encounter a high level of bureaucracy in their contracts, experience might induce exit. However, if firms perceive returns to experience, we should see increased exit among non-experienced firms. In the latter case, the treatment effect of experience underestimates the true returns to experience, since firms in the market survive precisely because of the it.  The OLS estimates underestimate the true effect of experience because we do not observe outcomes for firms that were unable to gain experience and had to abandon the market following defeats in the auctions. We briefly show in plot how exits disaggregated in terms of \% of firms that exit with and without experience per year.

A limitation of the analysis for the binary treatment is that it was only able to identify the Local Average Treatment Effect, which in the current context is interpreted as the treatment effect for those firms that can only acquire experience through a close win. Given our restrictive instruments' definition, this feature of the distribution of the causal effects is only applicable to a small part of our observations (between 2\% and 15\%, depending on the instrument). However, this is arguably the most important subsample, because in it there are firms that would achieve significant improvements after acquiring experience. Also, this discussion could show more evidence as to why we obtain higher IV than OLS estimates. Given the choice of the instrument, a firm that would only win in a close win should not have an absolute advantages in the market already, so it has more room to "grow".

The comparison of estimates for the treatment effect of experience between contracts that explicitly rewarded experience and those that did not (the main results) is relevant because it shows that the implicit effect of experience on outcomes is almost 60\% as the explicit effect. The explicit estimate of the treatment effect of any experience was around twelve percentage points, while for contracts that did not require it was seven percentage points. Given this, policymakers might prefer to only employ experience as a prequalification method, since it seems to largely keep operating in the case of no explicit reward for experience.

We found low $R^2$ in our regressions which shows that there is considerable heterogeneity in the outcomes. This can be attributed to the fact that we employed minimal types of controls in the regressions and wide array of types, locations, buyers and sellers. An alternative strategy would have been to i) add more controls or ii) consider a more restrictive market. Option i) was not employed because the sample is unbalanced in many ways and also because we do not have detailed contract description variables that could have been employed as controls. Option ii) could be used to obtain a more precise estimate in a clearly defined subsample, like contracts that need prequalification in the Ministry of Housing or Public Works. However, for these two government units the information was either incomplete or there was not a clear way to distinguish more "restricted" contracts beyond size.

\section{Operational Mechanisms of Experience}

The mechanisms section's objective was to test hypothesis about the improvements caused by experience in treated firms. Two possibilities were examined: improvements in cost measures, measured by the level of standardized bids submitted; and  quality levels, measured by the rate of acceptance of offers in a stage of the procurement process that verifies fulfillment of formal and/or technical requirements in proposals.

Firstly, the hypothesis that experience causes reduction in cost measures was tested. It was found that bids of firms with more than zero experience were between three and four percentage points lower than those that did not have any. The average difference between lowest and second lowest firms is around seven percentage points, so the impact can be significant if there is a binary reward to the lowest bid submitted. In this investigation, unlike most of our tests, we found linear experience to not have a significant coefficient. This might because this analysis employed total experience, with no adaptations such as annualizing or considering shorter periods. Given that at the last observations we have firms with very high measures of experiences (>100 contracts) it is expected that due to diminishing returns a linear return on experience is not the best choice.

Is an improvement of three percentage points truly useful to win more contracts? The results on lower bid amounts were significant, but it could be argued that the wide amount of factors employed to award projects render the effect negligible. However, a quick regression of the winning outcome (0-1) of the auction (for each firm that submitted a proposal) on standardized bids, with the usual fixed effects (see Appendix for details) shows that for every ten less percentage point on bid amounts, winning probability increase by around 2 percentage points. Thus, there at least correlation between lowered bid submitted and winning probability.

The result that first entrants bid more aggressively than firms with more than one year in the market was in line with previous literature results. Notably, the net effect of experience and first entry shows that an experienced firm still submits lower bids on average than first entrants.

The second hypothesis examined was that experience improves the quality of the proposals that a firm submits for auctions. The acceptance rate of proposals in the formal check stage of the procurement process was employed as a quality measure. We found that firms with strictly positive experience have acceptance rates that are around ten percentage points higher than firms with no experience. This effect is relevant considering that the average rate of acceptance is around 80\%, so the effect of experience drives acceptance rates close to 90\%.

It could be argued that the effect observed corresponds only to an adaptation experienced naturally after participating in the first "trial" auctions and that it only comes from bidding instead of experience. However, the analogous treatment effect of $bidding \ experience$ on outcomes is less than the effect of experience (details on the Appendix). While there seems to be a component of the effect related to "knowing the market", the effect of experience goes above and beyond this.

A remark should be made regarding the assumption that improved acceptance rates are related to improved quality. The improved quality identified in the result should be interpreted narrowly here as a better consideration of formal requirements in the proposal. A reasonable assumption is that all quality aspects of a bid are correlated and then that this relates to overall improvements in quality measures for the firm.

Overall, we mostly discussed costs measures, bids and quality as evolving due to within-firm changes. In this context, increased winning rates and improved acceptance rates are "positive". However, a part of these outcomes could be related to rent-seeking and capture of the market, by knowing "tricks" that inexperienced firms do not, or even corruption. The existence of legal rules  and the employment of a digital platforms constructed to prevent communication or knowledge of bidders before the awarding decision should diminish the opportunity for these types of situations. Still, we cannot completely rule them out.

Another regrettable omission of the data is the lack of comprehensive data for the Ministry of Public Projects. While on absolute numbers the contracts of this government unit are less to the ones of municipalities, for example, because of their complexity and size they are expected to have high returns to experience and of more interest. This organism started publishing their data comprehensively only since 2017, so before that year the data is incomplete.

\section{Implications for the market}
The magnitude of the effects found for experience could work as an entry barrier for new entrants to public construction projects. However, the econometrical interpretation of our treatment effects allows only us to say that firms that get in the market because of "random" wins improve in their outcomes. In that sense, the results points towards experience as an entry barrier for firms without strong comparative advantages in the market \textit{ex ante}.

The effect on the competitiveness on the market (only considering the treatment found) is then to limit the rise of "bad" firms which would become "good" with some experience. While this would be probably an undesirable feature in a private market, it could be argued that public markets should be focused on procuring goods as efficiently as possible. Then, depending on the tolerance to the distortion of preventing some firms to develop in the market, a policymaker might not be troubled by the results observed.

A possible effect suggested by the results is double-counting when considering experience in the award criteria. As it was seen, almost 60\% of the contracts include experience in the awarding criteria. However, it was also seen that more experience contractors already display qualities that make them more likely to win projects, like lower cost measures and better proposals. Then, from a competitive perspective, it could be better to rise technical or economical requirements to award the project but diminish the experience requirement (which as it was discussed is also a noisy measure of skill). If experience is truly a desirable property, we expect its effects to manifest in other aspects of the proposal that will make experienced candidates more likely to be awarded the project anyways.

%This could be a desirable outcome, since we have shown that more experienced firms submit proposals with higher rates of acceptance and more economical.

\chapter{Conclusion}
The paper's objective was to understand the treatment effect of experience on outcomes in the market of public construction projects. The investigation analyzed around 150,000 bids from 43,000 calls for proposals to compare the rate at which firms with and without experience win contracts in the future. It also analyzed possible mechanisms that would explain improved outcomes for firms with experience: a diminution of firm's cost measures, as measured by standardized bids submitted; and improved proposal quality, as measured by the acceptance rate of firm's proposals in a stage of the procurement process where formal requirements are processed.

The results pointed towards significantly improved outcomes in the future for firms with previous experience. Experienced firms win more contracts, bid more aggressively, and submit better quality proposals (as measured by their acceptance rate). The identification strategy, although limited in scope, renders significant and mostly precise estimates of the relevant parameters, for a subset of the firms that acquire experience only in close contests.

This investigation is relevant to the literature in bidding, public purchases and industrial organization because of its wide scope and empirical findings. The data employed spans a whole country, most of the public purchases developed in the construction sector and more than ten years of data. Regarding the empirical contributions, this investigation adds a dynamic component to the static investigation of auction competition. Also, it treats experience as an endogenous variable, developing a causal analysis of the influence of experience on market outcomes, while the existing literature usually develops OLS regressions. The size of the sample allows to identify with precision a feature of the causal effects distribution, namely, the Local Average Treatment Effect. The same strategy allowed to gain insight into actual operational differences that give firms advantages, bringing together both economics and engineering analysis.

Given the sizable impacts identified for firms that acquire experience through close contracts, the results are relevant for policymakers aiming to improve the competitiveness of public markets and those looking to improve the design of public auctions.

Finally, the work could be complemented in the future by a general model of bidding in the public sector. The heterogeneity of outcomes that we found shows that this effort would require to improve in the characterization of both units and contracts to yield detailed results at the firm level. On a separate field, the variables constructed would be also useful to construct a machine learning approach to detect suspicious awarding decisions to improve overall procurement transparency and efficiency.

%The main empirical strategy was to slice the dataset in different points in time and compare outcomes for the different levels of the treatment measures. Also, close wins were employed as source of random variation to instrument total wins, since experience is probably endogenous in the market. Close wins were defined employing a price strategy, where close wins are the ones where bids were close and price was a major awarding factor; and rank strategy, where a ranking was constructed for every firm and thus identify contests among similar contracts. Although both were second-best strategies employed due to the lack of proposal's score variables, both alternatives, along with most of an extensive set of robustness results, yielded similar outcomes.
%\section{Data and Others Considerations}
%A contribution of this project was to employ a comprehensive set of data regarding public purchases with ten years of public purchases. The data contained a great deal of useful and detailed variables, which were key for the investigation and are rarely available in other similar investigations about bidding behavior. The most important advantage of the dataset is that it covered almost all government bodies, which allows to construct a measure of public experience that we believe is very close to the actual level of experience in open call for proposal auctions. Investigations conducted previously in the literature tend to rely on markets much more restricted in scope, considering for example in the U.S case usually road projects from the Department on Transportation.

%There are still ways to improve the data. It is hard to store systematically the scores of a proposal in every item, since they differ among contracts. Nonetheless, some elements, like experience, price and quantity items could be made standard fields and stored due to their prevalence. The most difficult part of data collection was to gather the data for the awarding criteria, since it is unstructured (not in a database) and gathering it required to employ a (capped) API and manually programming a scraper to crawl the required information. Fortunately, the awarding criteria was usually stored similarly in the relevant webpage for every contract (the difficult part is knowing the URL of the webpage). Still, we had around 10\%-15\% of contracts with missing data on these fields, which had to be dropped from outcome computation.
