%% This is an example first chapter.  You should put chapter/appendix that you
%% write into a separate file, and add a line \include{yourfilename} to
%% main.tex, where `yourfilename.tex' is the name of the chapter/appendix file.
%% You can process specific files by typing their names in at the
%% \files=
%% prompt when you run the file main.tex through LaTeX.
\chapter{Literature Review}

Three strands of literature are relevant to this project. First, the economic modelling of learning by doing on firms. Second, the effect of experience on bidding for public projects. Finally, literature about to causally identification of incumbent power in a wider array of settings.

The first study of the effect of learning by doing on market outcomes was done by \citep{fudenberg1983learning} . They analyze learning by doing in two competition settings: perfect competition and strategic interaction. They show that in the case of strategic interactions firms choose to produce more in a first period than they would in a competitive setting. Also, output may decrease over time. Policy recommendation is to tax output in first period and subsidize in second, without a net transfer. Building upon this analysis, \citep{dasgupta1988learning} examine the effect of gains from learning in market structure. They extend the previous approach, which assumed a symmetrical setting among firms, by allowing for heterogeneity at the start of the competition. The most important result is that in the presence of more efficient firms at the beginning, the accumulation of capital through learning by doing can lead to concentration or even monopoly. Another important result is that firms may tolerate losses in the first years in anticipation of future profits.

The literature that studies the returns of human capital constitutes a relevant starting point for the study of the returns of learning by doing among firms, which can be seen as the process and returns of organizational capital. One of the most important branches in the aforementioned literature is the study of return to education. Since ability is endogenous in educational decisions, the literature of returns to education i) develops a framework easily assimilable to the firm case and ii) has develop useful econometric techniques for the current investigation. The article by \citep{card2001estimating}, reviews the most important developments, including his own influential paper on \citep{card1993using}, and also reviews the process of causal interpretation of the estimates of the returns to human capital produced via Instrumental Variables strategies, which are employed in the current investigation. 

On a different perspective, \citep{fu2002effect} study the effect of experience on contractors and set formal grounds to define and measure experience. Then they investigate if more experienced construction contractors are more aggressive in their bids for public projects. They examine the effect of bidding experience and past contracts won on bidding competitiveness. They find that more experience in bidding (not in contracts developed) leads to more aggressive bidding in building projects (more complex contracts), but not in renovations projects (simpler contracts). The amount of bidding aggressiveness depends also on the competition firms face, as pairings with similar experience backgrounds do not show increased levels of competition.

In a similar investigation, \citep{li2012construction} test if entrant subcontractors bid more aggressively than standard subcontractors in a platform of subcontractor hiring by construction firms. They test i) if entry bidders show more variance in their bids at the time of entrance and ii) if entry bidders bid lower than established players. They find that new entrants bid more aggressively than established players. However, they also find that historically higher bid winning ratios are associated with more aggressive bidding, which they attribute to cost variables or appetite for risk. Among the factors they hypothesize could explain more aggressive bidding by entry players is the necessity to establish a foothold in the market and the desire to avoid the winners curse by established players.

A similar pattern is found in \citep{estache2010bidder}, where fringe (weak) bidders are shown to be more aggressive in their bids when facing incumbents. This paper is concerned with similar questions that of the current investigation, although it is mostly focused on costs. However, as its empirical strategy performs only OLS regression, the paper recognizes that negative coefficients of experience on bids are mostly the symptom of endogenous selection, not a causal relationship.

Finally, an important source of methodological techniques for the investigation of incumbent strength is \citep{lee2004voters}. In this paper, the authors investigate the effect of election outcomes on political actions. A step in their analysis is accounting for incumbent power in the estimation analysis. They exploit closely won electoral races as quasi experiment to introduce random variation and produce results that are essentially arising from random noise. Their setting, which employs races which were decided by small margins, can also be employed to identify auctions won essentially by chance, in which cost variables did not provide the decisive advantage.
