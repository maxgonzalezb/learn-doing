%% This is an example first chapter.  You should put chapter/appendix that you
%% write into a separate file, and add a line \include{yourfilename} to
%% main.tex, where `yourfilename.tex' is the name of the chapter/appendix file.
%% You can process specific files by typing their names in at the
%% \files=
%% prompt when you run the file main.tex through LaTeX.
\chapter{Institutional context}
\section{Procurement and Public purchases in Chile}
\subsection{Public purchases via open call for proposals}
In general, all government units develop auctions to procure differentiated and non-standard goods and services (undifferentiated products, like office materials, are instead developed by a different type of method called framework agreement).  Government units usually advertise the project with an open call for proposals, receive tenders by interested firms and then award the project by ranking proposals with a weighted scoring method. In what follows we describe the auctioning process, awarding methods, some exceptions to the general rule, and legal requirements for contractors to participate in the market.

Usually, auctions have the following stages. First, the government sets up an open call for proposals for a specific project in a digital platform called Mercado Público, making avalaible relevant documents about the requirements for the project and detailing the awarding criteria that will be employed. Firms submit their tenders through the same digital platform, but cannot see tenders submitted by other firms. During the open call phase, firms can submit questions to the government, which, along with the government answer, are published online. When the tendering period ends, the analysis of proposals is done in two steps. First, government officials examine all proposals and ensure that they fulfill the minimum formal requirements to be evaluated on an equal footing with other proposals. All the proposals that fulfill the formal requirements are considered “Accepted”. The second step is to score all the “Accepted” proposals in terms of the awarding criteria and rank them. The top proposal is selected and awarded the project or service.

For each project the government chooses a set of items in whcih proposals will be evaluated on and a corresponding weight, which sum up to 100\%. The most frequent awarding items include price, technical specifications, quality, experience, etc. At the second awarding sub-stage, each proposal is given a given a score on each item,  based on rules specified in the tendering documents. Individual item's scores and multiplied by the corresponding weight and then summed up. The proposal's score is this weighted sum. The proposal with the highest score is awarded the project.

Before the call for offers, the auctioneer must establish an estimate of the total cost of the project. If the winning proposals is above 30\% of this estimate, the government unit must justify thoroughly the reasons that justify this disparity and keep additional information of the contract for further revisions.

The government unit in charge of tendering can employ two alternative procurement methods to an open call for proposals. It can develop a private auction (where only a subset of contractors are invited to submit proposals) or award directly the project to a contractor of its choice. However, there are several legal requirements for a project to be eligible for these types of procurement methods. Some cases where were direct or private auctioning is permitted a very specific product is required (so there is only one or a few providers) or the project is an extreme region where there are too few providers. This type of awarding method usually receives more scrutiny from the Contraloria, the government unit which checks if government actions are carried out within the appropriate legal rules.

All companies must register as public contractors in order to bid for public projects in a registry called Mercado Público. The purpose of this registry is to ensure that contractors are in good legal standing, and that they have no outstanding debts with the government treasury. It allows to keep a track record of past perfomance in government contracting for every contractor. The registry is also useful to identify potential conflicts of interest between firms' executives or firms' owners with government officials, as firms must disclose their ownership scheme at the time of registering. Even though every contractor must fulfill the same minimumal requirements in this registry, some government units, like the Ministry of Housing, maintain additional registers focused on the specific projects that the unit develops, These registries usually include additional requirements, classify contractors into categories according to their expertise and financial capacity.
%These are discussed in the detailed section about types of projects.

\subsection{Procurement And Information}
In Chile, as a general rule all government bodies must develop procurement procedures through a digital platform called the Mercado Público ( \textit{Public Market}).  This obligation was introduced by the Public Purchases law N° 19.886 (2010) and requires from government units to develop all stages of the process only through the platforms established by the Directorate of Public Purchases, more commonly known as Chile Compra, dependent from the Ministry of Treasury.

Although in the public construction sector different types of projects have different rules for how to conduct the details of the procurement process, the law mentioned above still requires from every government unit developing purchases to publish a common set of information to the digital platform. Some exceptions apply: contracts subject to considerations of national security, cases where providers cannot use the digital systems, and other considerations of force majeure. Among the information that the law requires to publish is the date of the auctions, any modifications to the blueprints, and the awarding decision.

The data of projects developed via Mercado Publico has been made public through an open data platform, which is the primary source of our data.

%Usually, procurement can proceed through framework arrangements or project-specific auctions Framework arrangements are auctions held by the government ex-ante where firms compete to be included in menu of similar products, usually with low to medium degrees of differentiation. If they win, their product can be directly bought by a government unit without the need for a separate auction. The framework agreements are usually employed to procure simple, less-differentiated products such as office materials, notebooks, etc. In the current work, we do not include data of products bough in this way in our analysis because we are not interested in materials or standard services but rather in full construction projects.

\section{Procurement of Construction Projects}
Tha law 19,886 and its procedures for procurement, detailed in the previous section, regulates public purchases in general. However, it excludes from its application contracts of public works. A portion of the contracts found in our dataset fall into this definition \footnote{Not all construction works are considered public works}. In this section, we briefly detail what commonly distinguishes construction procurement from regular government purchases, what are the common features among construction procurement regulation, and what are the differences among them.

Rquirements for contractors are usually increased in construction contracts to mitigate the possibility of adverse selection. We note two factors that increase requirements for firms in construction projects. First, capital avalaibility requirements, as many units include in the awarding criteria measures of equity to reduce the probability of bankruptcy or no access to credit during the project. Second, many construction projects require a bond that can be between 3-10\% of the total value of the project from the contractor to insure against problems during delivery.

Among construction projects with different applicable regulation, we usually see as common features of the procurement and awarding process an open call for proposals and a two stage awarding process. The first stage examines formal and technical requirements and the second assigns scores in the awarding criteria of the project. Differences among construction projects' regulation relate to the requirements to contractors to participate in auctions, the types of criteria that can be used to award the project, and the degree of discretion that can be employed in the process in general. Increased levels of contractor requirement or less discretionary processes are usually linked to more complex or bigger projects.

For example, certain types of projects requires prequalification steps and registering in a unit-specific registry which ensures financial capacity, experience, and skills.

Finally, even if a contract has its own set of applicable regulation, the Law of Public Purchases states that its own set own set of regulations shall be applicable wherever it is not contradictory with the more specific regulation.
%Altough the construction projects in our dataset are construction contracts, only a portion of them fall under the category of public works and would thus be excluded from its application.

%We now describe the particularities of the institutional framework for the procurement of construction projects. First, we advance some common qualitative differences, and given the heterogeneity of the projects in the dataset we next detail the relevant context for each type of project available in it.

\subsection{Institutional framework of types of construction projects found in the dataset}
Our sample is obtained by extracting observations which belong to the category of "construction projects" in a dataset that contains public purchases from a much wider array of categories of products. Since the classification is related to the type of product rather than the legal framework applicable, our dataset is heterogeneous in both the types of projects included and the institutional context relevant to each. We now get into more detail about the types of projects and the legal and institutional framework that aplies to each one, focusing on the rules about the procurement process relevant to the current investigation.

Since we do not have a single variable in the dataset that describes the framework applicable or the type of the project, the description is based on examination of the units and of the descriptions of the projects found most commonly in the dataset. Consequently, what follows may not constitute an exhaustive enumeration of the types of projects.

\begin{enumerate}[wide, labelwidth=!,labelindent=0pt,label=\textbf{\arabic*}.]
\item \textbf{Small maintenance and construction projects}:
All government units require infrastructure to operate, in the form of offices or facilities. As such, they regularly need to procure small and medium contracts to improve or develop maintenance work on the buildings that they employ. In these cases, projects are usually directly procured by the unit interested in it under the legal framework of the Law of Public Purchases detailed before.

We also consider under this category projects that improve or renovate facilities employed to deliver public services, like public schools, hospitals, communal health services, etc. If the project is relatively small and simple, so that it does not require specialized technical capabilities to evaluate proposals, it will fall under the same regulations stated above for purchases in general.

\item \textbf{Urban works}: projects in this category are works destined to maintain, improve or build public spaces like parks, streets, etc. Both municipalities and SERVIUs (detailed below) can procure these types of projects.

Municipalities can procure small construction works of communal development to attend to urban necessities, as specified in the law 18,695. These types of projects are usually low-to mid-size and subject to the procurement process specified in the Law of Public Purchases and the discretion of the Municipality. Examples of projects of this type found in the dataset are the construction and maintenance of parks, public graveyards, and communal meeting houses.

Urban works can also be procured by SERVIU as stated in the law. SERVIUs are the regional branches of the Ministry of Housing and Urbanism dedicated to execute projects in the areas of urban improvement, housing and drainage. Compared to urban municipalities's projects, SERVIUs develops usually bigger and more complex projects. The procurement regulations of SERVIUS are contained in the decree N°236 and , . It is stated that projects must be procured via an open call for proposals and warding can be done employing one or more criteria, giving emphasis to i) the quality of the project and ii) cost measures. The awarding process consists in two stages, where the first stage checks the inclusion of all required documents, and the second stage scores and ranks accepted proposals in the evaluation criteria of the project.

To participate in any call for proposals, interested firms must first register in a special registry maintained by MINVU, called RENAC. .

\item \textbf{Urban Road Projects:} the dataset contains road projects executed within urban limits. These projects can be developed by Municipalities as stated in Law 18,695 or by SERVIUs as stated in . Many times, the projects are executed in a collaboration between the two government units.

If the project is developed by the Municipality, it can employ its own set of directives following framework of the Public Purchases Law. If the project is developed by a SERVIU, it must abide by the rules detailed in the first item for this government unit.

\item \textbf{Housing Projects:} SERVIUs execute housing projects to achieve the objectives defined by the social habitatational policy of the Ministry of Housing and Urbanism. A key component of of this policy is the Social Integration program, which since 2015 has executed more than 127,000 social houses in zones with good services and transport avalaible. The rules for procurement are similar to other SERVIU projects.

By virtue of law 18,138, Municipalities can also develop social housing projects, which must be targeted at unfavoured sectors of their territory. The law mandates to employ an open call for proposals except in qualified cases like small sized projects and little time avalaible, when the Municipality can employ direct award methods. Municipalities can also execute sewage projects to completement housing projects or on their own.

\item \textbf{Water Drainage Projects:} law 19,525 assigns to SERVIUs the construction of part of the rainwater drainage network. The contracts of this type are subject to the same set of procurement regulations as other SERVIU contracts (i.e registry, opening in two steps, and awarding decision).

\item \textbf{Construction of Government Buildings:} we consider under this category the execution of complex buildings and facilities employed in the provision of public services, such as health, education, etc. It also includes the construction of facilities for the functioning of the different government units.

Since most government units do not have the technical expertise to carry out a full procurement and delivery process for complex projects, they can mandate another government unit, with specialized construction knowledge, the execution of the tendering and delivery of the project. A common alternative is to delegate the project to the oversight of the Architecture Directorate (Dirección de Arquitectura) of the Ministry of Public Works. The projects procured via the Architecture Directorate should be considerably less than the projects procured directly by the government units as the former is reserved for projects of increased complexity and size.  For example, in the case of hospitals, the Ministry of Health is in charge of defining the required hospital projects and the technical requirements. However, it signs agreements with the Ministry of Public Works delegating execution of the project to the Architecture Directorate.

The contracts procured by the Architectural Directorate, and in general all contracts from the Ministry of Public Works, are regulated by the Decree N° 75. In its first article, it mandates that contracts will be procured employing an open call for proposals. Proposals are evaluated in two stages. The first stage is a technical evaluation which verifies the inclusion in the proposal of technical requisites specified in the project. Proposals that do not fulfill this requirement are rejected and discarded. The second stage is the economic evaluation. The economic evaluation considers only price as awarding criteria for some types of contracts, and in these projects the project is awarded to the most economic proposal. For other types of contracts, the project is awarded taking into consideration the project, the experience of the contractor, and the execution plan. The evaluation proceeds by making discounts to the final price offered by the contractor for the project based on positive evaluations of these items (article 14°, decree N° 109). Then, after discounts, the project is awarded to the most economic proposal.

To participate in an auction from the Architecture Directorate, firms must be registed in the Contractor's Registry of the Ministry of Public Works, which imposes prrequisites on financial capacity, experience, and skills of the technical staff of the firm.

\end{enumerate}
%We end up detailing projects not found in the dataset.

%For example, in the case of hospitals, the Ministry of  Health is in charge of defining the projects, technical requirements, etc. of the hospitals that will be built. However, it usually signs agreements with the Ministry of Public Works where the Ministry of Public Works either i) directly oversees the auction and execution of the project via the Architecture Directorate or ii) develops a Public-Private Partnership to award to a contractor that will design, build, and operate the project. The last of these agreements was done in 2018 and it specified that 7 hospitals would be directly overseen by the Architectural Directorate and 18 would be developed via PPPs.

%The first way is to directly procure smaller projects (repairments, small building works, etc.) with no support. This is mostly employed by municipalities developing small urban projects, or by other government units developing small repair and maintenance projects. These types of projects fall directly under the regulations of the 19.86 law and the specific regulations of each procuring body.

%Some construction projects is that they require from contractors to be registered in a special Registry, which has specific experience, capital, and other requirements. Two of the government units that employ this special registry is the Ministry of Housing and the Ministry of Public Works.

%Although the previous elements describe some qualitative differences commonly found, project institutional context can differ highly depending on the specific type of the project. The following section details the different types of projects found in the dataset and their main institutional frameworks.
%The importance of considering the particularities of the construction sector is that several of the specific arrangements for these projects imply higher entry barriers to new contractors, which effect could be added to an eventual effect of experience on outcomes, which is the focus of the investigation.

%In some types of projects, for example, urban road projects, municipalities usually associate with the Ministry of Housing and Urbanism (MINVU) for financial or technical support. In rural areas, urban works developed by the Municipality include sewer and potable water infrastructure, in which they are supported by the Ministry of Public Works (MOP) or MINVU. In all these cases, the relevant legal framework will have additional requirements.
%\subsection{Summary}
%Projects in our investigation have different types of regulations applicable to them depending on the type, scope and size of the project. In general, we find the following common feature in the procurement process: an open call for proposals, a two stage awarding process, where the first stage examines formal and technical requirements and the second assigns scores in the awarding criteria of the project.With respect to other public purchases, requirements for contractors are usually increased in construction contracts to mitigate the possibility of adverse selection. We note two factors that increase requirements for firms in construction projects. First, capital avalaibility requirements, as many units include in the awarding criteria measures of patrimonio to reduce the probability fof bankruptcy or no access to credit during the project. Second, surety bonds, as many construction projects require a bond that can be between 3-10\% of the total value of the project from the contractor to insure against problems during delivery.Differences among construction projects' frameworks relate to the requirements to contractors to participate in auctions. More complex and certain types of projects requires prequalification steps and registering in a unit specific- which ensures financial capacity, experience, and skills.

%Finally, even if a contract has its own set of applicable regulation, the Law of Public Purchases states that in everything that is not explicititly changed in the specific regulation, it shall be applicable.
