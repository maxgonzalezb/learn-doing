%% This is an example first chapter.  You should put chapter/appendix that you
%% write into a separate file, and add a line \include{yourfilename} to
%% main.tex, where `yourfilename.tex' is the name of the chapter/appendix file.
%% You can process specific files by typing their names in at the
%% \files=
%% prompt when you run the file main.tex through LaTeX.
\chapter{Institutional context}
\section{Procurement and Public purchases in Chile}
\subsection{Open call for proposals and public purchases}
In general, all government units develop auctions to procure differentiated and non-standard goods and services (undifferentiated products, like office materials, are instead developed by a different type of method called framework agreement).  These auctions usually advertise the project with an open call for proposals, receive tenders by interested firms and award the project by ranking proposals with a weighted scoring method. In what follows we describe the general auctioning process, awarding method, some exceptions to the general rule, and contractor’s requirements.

Usually, auctions have the following steps. The auction begins with an open call for proposals from different firms, who submit their tenders through a digital platform called Mercado Público. Firms cannot see tenders submitted by other firms. During the open call phase, firms can submit questions to the government, which, along with the government answer, are published online. When the bidding period ends, the analysis of tenders is done in two steps. First, government officials examine all proposals and ensure that they fulfill the government’s requirements in terms of formal requirements (e.g., present all legally requested documents). All the proposals that fulfill the formal requirements are considered “Accepted”. The second step is to score all the “Accepted” proposals in terms of the awarding criteria and rank them. The top proposal is selected and awarded the good or service.

Each auction is associated with a unique set of i) awarding criteria and awarding criteria’s weight. Awarding criteria are the individual items by which the proposals are ranked. These items almost always include price, but which can also include technical specifications, quality, experience, etc. Each item also has an associated weight which is used to compute each firm’s final proposal score. For each item in the set, each proposal is given a given a score based on rules specified in the tendering documents and multiplied by the corresponding pre-specified weighted. The proposal with the highest score is awarded the project.
Before the call for offers, the auctioneer must establish an estimate of the total cost of the project. If the winning proposals is above 30\% of this estimate, the government unit must justify thoroughly the reasons that justify this disparity and keep additional information of the contract for further revisions.

The government unit in charge of tendering can employ two alternative procurement methods to an open call for proposals. It can develop a private auction (where only a subset of contractors are invited to submit proposals) or award directly the project to a contractor of its choice. However, there are several legal requirements for a project to be eligible for these types of procurement methods. Some cases where this is permitted is to acquire a very specific product (so there is only one or a few providers) or the project is an extreme region where there are too few providers. This type of awarding method usually receives more scrutiny from the Contraloria, the government unit which checks if government actions are carried out within the appropriate law rules.

All companies must register as public contractors in order to bid for public projects, in a registry called Mercado Público. The purpose of this step is to ensure that contractors is in good legal standing, and it has no outstanding debts. It also allows to keep a track record of contracting history with the government, to follow contractor performance. The registry is also useful to identify potential conflicts of interest between firms’ executives or firms’ owners with government officials, as firms must disclose their ownership scheme at the time of register. Even though there is a general register of contractors for all public purchases, some government units, like the Ministry of Housing, additional unit-specific registers tailored for the purposes of the specific projects that that unit develops, which have additional requirements and categories. These are discussed in the detailed section about types of projects.

\subsection{Procurement And Information}
In Chile, generally all government bodies must develop procurement procedures through a digital platform called the Public Market (Mercado Público).  This obligation was introduced by the Public Purchases law N° 19.886(2010) and requires from government units to develop all stages of the process only through the platforms established by the Directorate of Public Purchases, more commonly known as Chile Purchases (Chile Compra), dependent from the Ministry of Treasury.

Although in the public construction sector different types of projects have different rules for how to conduct the details of the procurement process, the law mentioned above still requires from every government unit developing purchases to publish a common set of information to the digital platform. Some exceptions apply: contracts subject to considerations of national security, cases where providers cannot use the digital systems, and other considerations of force majeure. Among the information that the law requires to publish is the date of the auctions, any modifications to the blueprints, and the awarding decision.

The data of projects developed via Mercado Publico has been made public through an open data platform, which is the primary source of our data.
Usually, procurement can proceed through framework arrangements or project-specific auctions Framework arrangements are auctions held by the government ex-ante where firms compete to be included in menu of similar products, usually with low to medium degrees of differentiation. If they win, their product can be directly bought by a government unit without the need for a separate auction. The framework agreements are usually employed to procure simple, less-differentiated products such as office materials, notebooks, etc. In the current work, we do not include data of products bough in this way in our analysis because we are not interested in materials or standard services but rather in full construction projects.

\section{Procurement of Construction Project}
We now describe the particularities of the institutional framework for the procurement of construction projects. First, we advance some common qualitative differences, and given the heterogeneity of the projects in the dataset we detail the relevant context for each type of project available in it. The importance of considering the particularities of the construction sector is that several of the specific arrangements for these projects imply higher entry barriers to new contractors, which effect could be added to an eventual effect of experience on outcomes, which is the focus of the investigation.

First, commonly the procurement process of public construction projects has unique features related increased requirements to contractors, whether in the awarding criteria or to submit proposals, to mitigate the risk that these projects involve. We note three factors that increase requirements for firms: capital requirements for contractors, surety bonds, and prequalification requisites. First, many construction projects include in the awarding criteria an item related to the existence of enough capital to develop the project without the danger of incurring in financial bankruptcy or losing access to credit. Second, many construction projects require surety bonds from firms. Construction projects usually require a bond that can be between 3-10\% of the total value of the project from the contractor. This ensures that the government is insured against adverse selection.

Finally, another feature of some construction projects is that they require from contractors to be registered in a special Registry, which has specific experience, capital, and other requirements. Two of the government units that employ this special registry is the Ministry of Housing and the Ministry of Public Works.
Although the previous elements describe some qualitative differences commonly found, project institutional context can differ highly depending on the specific type of the project. The following section details the different types of projects found in the dataset and their main institutional frameworks.

\subsection{Types of Construction projects in the dataset}
Municipalities’ Urban Works: municipalities can procure small construction works of communal development to attend to urban necessities. These types of projects are usually low-to mid-size and are in general subject to the procurement process specified in the Law of Public Purchases, detailed above.  The execution of these works is overseen by the Direction of Municipal Works. Also, projects included in this category can be small maintenance projects developed in bigger infrastructures. For example, annual school maintenance projects.
Examples of projects of this type found in the dataset are the construction and maintenance of parks, public graveyards, communal meeting houses.

In some cases, for example, urban road projects, municipalities usually associate with the Ministry of Housing and Urbanism (MINVU) for financial or technical support. In rural areas, urban works developed by the Municipality include sewer and potable water infrastructure, in which they are supported by the Ministry of Public Works (MOP) or MINVU. In all these cases, the relevant legal framework will have additional requirements.

Housing Projects: housing projects are aimed at generating a supply of low-cost houses so low-income workers can acquire their own houses. These projects can be developed by either the Ministry of Housing and Urbanism or by Municipalities, although big projects will usually be developed by the Ministry. Housing projects are governed by a heavy set of additional regulations, which include a set of specific procurement rules. The Ministry of Housing also has their own Registry of Contractors (RENAC) with its own set of pre-qualification requirements for firms to participate in these types of projects.

Government Buildings: government buildings include all the buildings intended to be employed in the provision of government services, such as health, education, etc. It also includes the construction of facilities for the functioning of the different government units. The projects in this category have all different sets of regulation depending on the sector and size. However, in broad terms they can be procured under three main frameworks.

The first way is to directly procure smaller projects (repairments, small building works, etc.) with no support. This is mostly employed by municipalities developing small urban projects, or by other government units developing small repair and maintenance projects. These types of projects fall directly under the regulations of the 19.86 law and the specific regulations of each procuring body.

The second alternative is to delegate the procuring process unto the oversight of the Architectural Directorate (Dirección de Arquitectura) of the Ministry of Public Works. The projects procured via the Architecture Directorate should be considerably less than the projects procured directly by the government units as the former is reserved for projects of increased complexity and size, usually full buildings from the ground, like a new hospital or school. Among these projects, the projects that i) have exceptional complexity or size ii) have potential users that pay for the use of infrastructure can be procured by Public- Private Partnerships (PPPs), the third procurement method. Within a specific project type, the proportion of projects that have been developed be each procurement scheme has varied in the last years with the political party in government, the budget available, and the organizational capacity of the government units. In order to develop a PPP, the government unit interested in a specific building requirement delegates the procurement of the project to the Directorate of Concessions of the Ministry of Public Works, who develops a Private Public Partnership initiative, in which a contractor is selected to design, build, and operate the project.
For example, in the case of hospitals, the Ministry of  Health is in charge of defining the projects, technical requirements, etc. of the hospitals that will be built. However, it usually signs agreements with the Ministry of Public Works where the Ministry of Public Works either i) directly oversees the auction and execution of the project via the Architecture Directorate or ii) develops a Public-Private Partnership to award to a contractor that will design, build, and operate the project. The last of these agreements was done in 2018 and it specified that 7 hospitals would be directly overseen by the Architectural Directorate and 18 would be developed via PPPs.
