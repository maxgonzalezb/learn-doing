

\centering
\textbf{Abstract}
\justify
\footnotesize
Using 43,000 public construction contracts in Chile procured employing open calls for proposals, I study the effect of firm experience on the likelihood of winning a contract in the future. To address endogeneity of experience (better firms tend to win more contracts in the past and in the future), I instrument firm experience with the number of past contracts won in closely contested auctions, where close auctions are defined as either i) having close monetary bids and price as an important awarding factor ii) involving closely ranked firms (via a modified ELO algorithm) . The IV estimates indicate that firm experience increases the proportion of contracts won by seven percentage points (roughly a third of the winning rate of firms with no experience). I investigate possible mechanisms that could explain this increase in market success by improvements along i) cost measures and ii) quality variables. I find that experienced firms submit bids which are three percentage points lower than firms with no experience, which is correlated with an increase in winning probability. Additionally, experienced firms increase in ten percentage points the approval rate of their proposals in the first stage of the awarding process. I discuss the magnitude of the findings and possible implications for public auction design.
